\begin{frame}[t,plain]
\titlepage
\end{frame}

% a hack, since colons do not render in the math environments
\renewcommand{\colon}{\mbox{~:~}}

\begin{frame}[fragile]
    \frametitle{Main Goal}
  
    There is no set-theoretic model for System F, in which types denote sets, and $S \rightarrow S'$ denotes the set of all functions from $S$ to $S'$.
\end{frame}

\begin{frame}{Overview}
\tableofcontents
\end{frame}

\section{Preliminaries}

\subsection{System F}

\begin{frame}[fragile]
    \frametitle{System F}
  
    The
    \begin{itemize}
        \item \textbf{System F}
        \item \textbf{Girard--Reynolds polymorphic lambda calculus}
        \item \textbf{polymorphic lambda calculus}
        \item \textbf{second-order lambda calculus}
    \end{itemize}
    is an extension of typed lambda calculus, in which types can be passed to functions as arguments.
  
    $$
    \begin{matrix}
    \infer{\Gamma \vdash M\alpha \colon \omega[\alpha/\tau]}{\Gamma \vdash M \colon \forall\tau.\omega}
    \ &\ 
    \infer{\Gamma \vdash \Lambda\tau. M \colon \forall\tau.\omega}{\Gamma \vdash M \colon \omega}
    \end{matrix}
    $$
    \pause
    Or in another notation \cite{REYNOLDS19931}:
    $$
    \begin{matrix}
    \infer{\pi \vdash_N e[\alpha] \colon (\omega / \tau \rightarrow \alpha)}{\pi \vdash_N e \colon \Delta\tau.\omega}
    \ &\ 
    \infer{\pi \vdash_N \Lambda\tau. e \colon \Delta\tau.\omega}{\pi \vdash_{N \cup \{\tau\}} e \colon \omega}
    \end{matrix}
    $$
\end{frame}

\begin{frame}[fragile]
    \frametitle{Normality in System F}
  
    \begin{itemize}
        \item In System F, a term is called a \textit{normal}, if it does not contain any terms of the form $(\lambda x_\alpha. \omega)(a)$ or $(\Lambda \tau. \omega)[\alpha]$  \cite{capretta_valentini_1999}.
        \item The following rules are called $\beta$-contractions:
        \begin{align*}
            (\lambda x_\alpha. \omega)(a) &\rightsquigarrow (\omega / x \rightarrow a) \\
            (\Lambda \tau. \omega)[\alpha] &\rightsquigarrow (\omega / \tau \rightarrow \alpha)
        \end{align*}
        \item A term is called \textit{normalizable}, if it can be reduced into a normal term through $\beta$-contractions.
        \item A term is called \textit{strongly normalizable}, if it can be normalized in finite contractions.
    \end{itemize}
\end{frame}

\subsection{Reynolds' Conjecture}

\begin{frame}[fragile]
    \frametitle{Reynolds' Conjecture}
  
    It is known that:
    \begin{itemize}
        \item All expressions of System F are normalizable \cite{girard1972phd}.
        \item All expressions of System F are strongly normalizable \cite{PRAWITZ1971235}.
        \item The elements of any free many-sorted anarchic algebra are isomorphic to the closed normal expressions of a type that is determined by the signature of the algebra \cite{BOHM1985135}.
    \end{itemize}

    These facts led to the following conjecture:
    \begin{block}{Reynolds' Conjecture}
		System F has a set-theoretic model in which types denote sets and $S \rightarrow S'$ denotes the set of all functions from $S$ to $S'$ \cite{reynolds1983types}.
	\end{block}
\end{frame}

\section{History}

\begin{frame}[fragile]
    \frametitle{History}
  
    \begin{itemize}
        \item Reynolds later proved that no such model exists \cite{reynold-not-set-theoretic}.
        \item Shortly thereafter, Plotkin generalized this proof by considering, for models based upon arbitrary Cartesian closed categories, the behavior of functors that can be defined in the calculus \cite{plotkin-non-existing-article}.
        \item Later, in a joint paper, they gave an exposition of this generalization, and showed why it precludes the existence of several kinds of model \cite{REYNOLDS19931}.
    \end{itemize}
\end{frame}

\section{Formal Definition of System F}

\begin{frame}[allowframebreaks]
    \frametitle{Formal Definition of System F}

    The languages is built from:
    \begin{itemize}
        \item $\mathcal{T}$ infinite set of \textit{type variables},
        \item $\mathcal{V}$ infinite set of \textit{ordinary variables},
        \item $\Omega_N$ of \textit{type expressions} over a finite set $N$ of type variables,
        \item $E_V^N$ of \textit{ordinary expressions} over finite sets $N$ of type variables and $V$ of ordinary variables,
        \item $\Omega^*_N$ of \textit{type assignments} over $N$, i.e. functions $\pi \colon \mathrm{dom}(\pi) \rightarrow \Omega_N$,
        \item \textit{typing} relation $\pi \vdash_N e \colon \omega$, for $\pi \in \Omega^*_N$, $\omega \in \Omega_N$, and $e \in E^N_{\mathrm{dom}(\pi)}$.
    \end{itemize}
    
    \framebreak
    \begin{enumerate}
        \item If $\tau \in N$, then $\tau \in \Omega_N$.
        \item If $\omega, \omega' \in N$, then $\omega \rightarrow \omega' \in \Omega_N$.
        \item If $\tau \in \mathcal{T}$ and $\omega \in \Omega_{N \cup \{\tau\}}$, then $\Delta\tau. \omega \in \Omega_N$.
        \item If $N \subseteq N'$, then $\Omega_N \subseteq \Omega_{N'}$.
        \item If $N \subseteq N'$, then $\Omega^*_N \subseteq \Omega^*_{N'}$.
        \item If $v \in V$ then $v \in E_V^N$.
        \item If $e, e' \in E_V^N$, then $ee' \in E_V^N$,
        \item If $v \in \mathcal{V}$, $\omega \in \Omega_N$, and $e \in E_{V \cup \{v\}}^N$, then $\lambda v_\omega. e \in E_V^N$.
        \item If $e \in E_V^N$ and $\omega \in \Omega_N$, then $e[\omega] \in E_V^N$.
        \item If $\tau \in \mathcal{T}$ and $e \in E_V^{N \cup \{\tau\}}$, then $\Lambda\tau. e \in E_V^N$.
        \item If $N \subseteq N'$ and $E \subseteq E'$, then $E_V^N \subseteq E_{V'}^{N'}$.
        \framebreak
        \item For $\pi \in \Omega^*_N$ and $v \in \mathrm{dom}(\pi)$,
        $
        \infer{\pi \vdash_N v \colon \pi v}{}
        $.
        \item For $\pi \in \Omega^*_N$ and $\omega, \omega' \in \Omega_N$,
        \[
        \begin{matrix}
        \infer{\pi \vdash_N e_1e_2 \colon \omega'}{
            \pi \vdash_N e_1 \colon \omega \rightarrow \omega'
            &
            \pi \vdash_N e_2 \colon \omega
        }
        ~&~
        \infer{\pi \vdash_N \lambda v_\omega. e \colon \omega \rightarrow \omega'}{[\pi~|~v\colon\omega] \vdash_N e \colon \omega'}.
        \end{matrix}
        \]
        \setcounter{enumi}{14}
        \item For $\pi \in \Omega^*_N$, $\omega \in \Omega_N$, and $\tau \in N$,
        \[
        \infer{\pi \vdash_N e[\tau] \colon \omega}{\pi \vdash_N e \colon \Delta\tau. \omega}.
        \]
        \item For $\pi \in \Omega^*_{N \setminus \{\tau\}}$ and $\omega \in \Omega_{N \cup \{\tau\}}$,
        \[
        \infer{\pi \vdash_N \Lambda\tau. e \colon \Delta\tau. \omega}{\pi \vdash_{N \cup \{\tau\}} e \colon \omega}.
        \]
        \item For $N \subseteq N'$, $pi \in \Omega^*_N$, and $\omega \in \Omega_N$,
        \[
        \infer{\pi \vdash_{N'} e \colon \omega}{\pi \vdash_N e \colon \omega}.
        \]
        \item For $\pi, \pi' \in \Omega^*_N$ such that $\pi = \pi'|_{\mathrm{dom}(\pi)}$, and $\omega \in \Omega_N$,
        \[
        \infer{\pi' \vdash_N e \colon \omega}{\pi \vdash_N e \colon \omega}.
        \]
    \end{enumerate}
    
    \begin{itemize}
        \item [15'.] For $\pi \in \Omega^*_N$, $\omega \in \Omega_{N \cup \{\tau\}}$, and $\omega' \in \Omega_N$,
        \[
        \infer{\pi \vdash_N e[\omega'] \colon (\omega / \tau \rightarrow \omega')}{\pi \vdash_N e \colon \Delta\tau.\omega}.
        \]
    \end{itemize}
\end{frame}

\section{Categorical Motivation}

\begin{frame}[fragile]
    \frametitle{Categorical Motivation}
    
    Given a model of System F based upon a Cartesian closed category $\mathcal{K}$, there will be a functors from $\mathcal{K}$ to $\mathcal{K}$ whose
    \begin{itemize}
        \item Actions on objects can be expressed by type expressions,
        \item Actions on morphisms can be expressed by ordinary expressions.
    \end{itemize}
    
    \vspace{1em}
    If $T$ is such a functor, then
    \begin{itemize}
        \item there is a weak initial $T$-algebra,
        \item if $\mathcal{K}$ possesses equalizers of all subsets of its morphism sets, there is a $T$-algebra.
    \end{itemize}
\end{frame}

\section{Mathematical Preliminaries}

\subsection{Category}

\begin{frame}[fragile]
    \frametitle{Category}
    
    When $\mathcal{C}$ is a category,
    \begin{itemize}
        \item $|\mathcal{C}|$ is the collection of objects of $\mathcal{C}$,
        \item $k \xrightarrow[\mathcal{C}]{} k'$ denotes the set of morphisms from $k \in |\mathcal{C}|$ to $k' \in |\mathcal{C}|$,
        \item $\mathcal{C}^{\mathrm{op}}$ denotes the dual of $\mathcal{C}$.
    \end{itemize}
\end{frame}

\subsection{Category Exponentiation}

\begin{frame}[fragile]
    \frametitle{Category Exponentiation}
    
    When $\mathcal{C}$ is a category which closed under finite products, and $k', k'' \in |\mathcal{C}|$, then an \textit{exponentiation} of $k''$ by $k'$ consists of an object $k' \xRightarrow[\mathcal{C}]{} k''$ and a morphism $\mathrm{ap}^\mathcal{C}_{k'k''} \in \left( k' \xRightarrow[\mathcal{C}]{} k'' \right) \times k' \rightarrow k''$ such that, for each $k \in |\mathcal{C}|$ and $\rho \in k \times k' \rightarrow k''$, there is a unique morphism, denoted by $\mathrm{ab}^\mathcal{C}_\rho$, in $k \rightarrow (k' \xRightarrow[\mathcal{C}]{} k'')$ such that
    \[
        \begin{tikzcd}[column sep = huge]
        k \times k' \arrow[r, "\mathrm{ab}^\mathcal{C} \rho \times I_{k'}"] \arrow[rd, "\rho"] & (k' \xRightarrow[\mathcal{C}]{} k'') \times k' \arrow[d, "\mathrm{ap}^\mathcal{C}_{k'k''}"] \\
        & k''
        \end{tikzcd}
    \]
    commutes in $\mathcal{C}$.
\end{frame}

\subsection{Cartesian Closed Categories and $\mathsf{SET}$}

\begin{frame}[fragile]
    \frametitle{Cartesian Closed Categories and $\mathbf{\mathsf{SET}}$}
    
    A category $\mathcal{C}$ is called Cartesian closed, iff
    \begin{itemize}
        \item It has a terminal object,
        \item If $k, k' \in |\mathcal{C}|$, then $k \times_\mathcal{C} k' \in |\mathcal{C}|$,
        \item If $k, k' \in |\mathcal{C}|$, then $k' \xRightarrow[\mathcal{C}]{} k = k^{k'} \in |\mathcal{C}|$.
    \end{itemize}
    
    \vspace{1em}
    As a special case, the category $\mathsf{SET}$ of all sets, with functions as morphisms, is Cartesian closed.
    \begin{itemize}
        \item Any singleton set is a terminal object of $\mathsf{SET}$.
        \item For sets $A, B \in |\mathsf{SET}|$, their Cartesian product $A \times B \in |\mathsf{SET}|$.
        \item For sets $A, C \in |\mathsf{SET}|$, the set of all functions from $C$ to $A$, $C \xrightarrow[\mathsf{SET}]{} A = A^C \in |\mathsf{SET}|$.
    \end{itemize}
\end{frame}

\subsection{The Functor $Q_c^\mathcal{C}$}

\begin{frame}[fragile]
    \frametitle{The Functor $Q_c^\mathcal{C}$}
    
    For any object $c$ of a Cartesian closed category $\mathcal{C}$, there is a functor $Q_c^\mathcal{C} \colon \mathcal{C} \rightarrow \mathcal{C}^{\mathrm{op}}$, such that $Q_c^\mathcal{C}(k) = k \xRightarrow[\mathcal{C}]{} c$ for all $k \in |\mathcal{C}|$.
\end{frame}

\section{Formal Definition of $\mathcal{K}$-Models}

\watermarkoff
\begin{frame}[fragile]
    \frametitle{Formal Definition of $\mathcal{K}$-Models}

    A $\mathcal{K}$-model of System F, consists of the following:
    \begin{enumerate}
        \item A Cartesian closed category $\mathcal{K}$.
        \item For each object assignment $O$ with domain $N$, a semantic function $\mathcal{M}O \colon \Omega_N \rightarrow |\mathcal{K}|$:
        \begin{enumerate}
            \item If $\tau \in N$, then $\mathcal{M}O\tau = O\tau$.
            \item If $\omega, \omega' \in \Omega_N$, then $\mathcal{M}O(\omega \rightarrow \omega') = \mathcal{M}O\omega \xRightarrow[\mathcal{K}]{} \mathcal{M}O\omega'$.
            \item If $O = O'|_N$ and $\omega \in \Omega_N$, then $\mathcal{M}O'\omega = \mathcal{M}O\omega$.
        \end{enumerate}
        \item For each object assignment $O$ with domain $N$, $\pi \in \Omega^*_N$, and $\omega \in \Omega_N$, a semantic function 
        \[
        \mu^O_{\pi\omega} \colon E^N_{\pi\omega} \rightarrow \left(\prod^\mathcal{K}\left(\mathcal{M}O \cdot \pi\right) \xrightarrow[\mathcal{K}]{} \mathcal{M}O\omega\right),
        \]
        where $\mathcal{M}O \cdot \pi \colon \mathrm{dom}(\pi) \rightarrow |\mathcal{K}|$ is defined by $(\mathcal{M}O \cdot \pi)v = \mathcal{M}O(\pi v)$.
    \end{enumerate}
\end{frame}

\section{Impossible Models}

\watermarkon
\begin{frame}[fragile]
    \frametitle{Impossible Models}
    
    \begin{block}{Theorem}
		Suppose $\mathcal{K}$ and $\mathcal{K}'$ are Cartesian closed categories such that $\mathcal{K}'$ is a sub-ccc of both $\mathcal{K}$ and $\mathsf{SET}$, there is an object $c$ of $\mathcal{K}'$ that contains more than one member, all objects and morphisms in the range of the functor $Q_c^\mathcal{K}$ belong to $\mathcal{K}'$, and all subsets of the morphism sets of $\mathcal{K}'$ have equalizers. Then there is no $\mathcal{K}$-model.
	\end{block}
	
	\begin{block}{Corollary}
	    There is no $\mathsf{SET}$-model.
	\end{block}
	
	\begin{proof}
	Simply, set $\mathcal{K} = \mathcal{K}' = \mathsf{SET}$.
	\end{proof}
\end{frame}


\begin{frame}[t,allowframebreaks]
\nocite{*}
\frametitle{References}
\bibliographystyle{apalike}
\bibliography{ref.bib}
\end{frame}
