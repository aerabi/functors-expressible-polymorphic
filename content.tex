\begin{frame}[t,plain]
\titlepage
\end{frame}

% a hack, since colons do not render in the math environments
\renewcommand{\colon}{\mbox{~:~}}

\begin{frame}[fragile]
    \frametitle{Main Goal}
  
    There is no set-theoretic model for System F, in which types denote sets, and $S \rightarrow S'$ denotes the set of all functions from $S$ to $S'$.
\end{frame}

\begin{frame}{Overview}
\tableofcontents
\end{frame}

\section{Preliminaries}

\subsection{System F}

\begin{frame}[fragile]
    \frametitle{System F}
  
    The
    \begin{itemize}
        \item \textbf{System F}
        \item \textbf{Girard--Reynolds polymorphic lambda calculus}
        \item \textbf{polymorphic lambda calculus}
        \item \textbf{second-order lambda calculus}
    \end{itemize}
    is an extension of typed lambda calculus, in which types can be passed to functions as arguments.
  
    $$
    \begin{matrix}
    \infer{\Gamma \vdash M\alpha \colon \omega[\alpha/\tau]}{\Gamma \vdash M \colon \forall\tau.\omega}
    \ &\ 
    \infer{\Gamma \vdash \Lambda\tau. M \colon \forall\tau.\omega}{\Gamma \vdash M \colon \omega}
    \end{matrix}
    $$
    \pause
    Or in another notation \cite{REYNOLDS19931}:
    $$
    \begin{matrix}
    \infer{\pi \vdash_N e[\alpha] \colon (\omega / \tau \rightarrow \alpha)}{\pi \vdash_N e \colon \Delta\tau.\omega}
    \ &\ 
    \infer{\pi \vdash_N \Lambda\tau. e \colon \Delta\tau.\omega}{\pi \vdash_{N \cup \{\tau\}} e \colon \omega}
    \end{matrix}
    $$
\end{frame}

\begin{frame}[fragile]
    \frametitle{Normality in System F}
  
    \begin{itemize}
        \item In System F, a term is called a \textbf{normal}, if it does not contain any terms of the form $(\lambda x_\alpha. \omega)(a)$ or $(\Lambda \tau. \omega)[\alpha]$  \cite{capretta_valentini_1999}.
        \item The following rules are called $\beta$-contractions:
        \begin{align*}
            (\lambda x_\alpha. \omega)(a) &\rightsquigarrow (\omega / x \rightarrow a) \\
            (\Lambda \tau. \omega)[\alpha] &\rightsquigarrow (\omega / \tau \rightarrow \alpha)
        \end{align*}
        \item A term is called \textbf{normalizable}, if it can be reduced into a normal term through $\beta$-contractions.
        \item A term is called \textbf{strongly normalizable}, if it can be normalized in finite contractions.
    \end{itemize}
\end{frame}

\subsection{Reynolds' Conjecture}

\begin{frame}[fragile]
    \frametitle{Reynolds' Conjecture}
  
    It is known that:
    \begin{itemize}
        \item All expressions of System F are normalizable \cite{girard1972phd}.
        \item All expressions of System F are strongly normalizable \cite{PRAWITZ1971235}.
        \item The elements of any free many-sorted anarchic algebra are isomorphic to the closed normal expressions of a type that is determined by the signature of the algebra \cite{BOHM1985135}.
    \end{itemize}

    These facts led to the following conjecture:
    \begin{block}{Reynolds' Conjecture}
		System F has a set-theoretic model in which types denote sets and $S \rightarrow S'$ denotes the set of all functions from $S$ to $S'$ \cite{reynolds1983types}.
	\end{block}
\end{frame}

\begin{frame}[fragile]
    \frametitle{History}
  
    \begin{itemize}
        \item Reynolds later proved that no such model exists \cite{reynold-not-set-theoretic}.
        \item Shortly thereafter, Plotkin generalized this proof by considering, for models based upon arbitrary Cartesian closed categories, the behavior of functors that can be defined in the calculus \cite{plotkin-non-existing-article}.
        \item Later, in a joint paper, they gave an exposition of this generalization, and showed why it precludes the existence of several kinds of model \cite{REYNOLDS19931}.
    \end{itemize}
\end{frame}

\begin{frame}[t,allowframebreaks]
\nocite{*}
\frametitle{References}
\bibliographystyle{apalike}
\bibliography{ref.bib}
\end{frame}
