
\section{Preliminaries}

\begin{frame}[fragile]
    \frametitle{Polymorphic Typed Lambda Calculus}
  
    The
    \begin{itemize}
        \item \textbf{System F}
        \item \textbf{Girard--Reynolds polymorphic lambda calculus}
        \item \textbf{polymorphic lambda calculus}
        \item \textbf{second-order lambda calculus}
    \end{itemize}
    is an extension of typed lambda calculus, in which types can be passed to functions as arguments.
  
    $$
    \begin{matrix}
    \infer{\Gamma \vdash M\alpha : b[\alpha/\xi]}{\Gamma \vdash M : \forall\xi.b}
    \ &\ 
    \infer{\Gamma \vdash \Lambda\xi. M : \forall\xi.b}{\Gamma \vdash M : b}
    \end{matrix}
    $$
\end{frame}

\begin{frame}[fragile]
    \frametitle{Normality in Polymorphic Typed Lambda Calculus}
  
    \begin{itemize}
        \item In System F, a term is called a \textbf{normal}, if it does not contain any terms of the form $(\lambda x: \alpha. b)(a)$ or $(\Lambda \xi. b)(\alpha)$  \cite{capretta_valentini_1999}.
        \item The following rules are called $\beta$-contractions:
        \begin{align*}
            (\lambda x: \alpha. b)(a) &\rightsquigarrow b [a/x] \\
            (\Lambda \xi. b)(\alpha) &\rightsquigarrow b [\alpha / \xi]
        \end{align*}
        \item A term is called \textbf{normalizable}, if it can be reduced into a normal term through $\beta$-contractions.
        \item A term is called \textbf{strongly normalizable}, if it can be normalized in finite contractions.
    \end{itemize}
\end{frame}

\begin{frame}[fragile]
    \frametitle{Normality in Polymorphic Typed Lambda Calculus}
  
    \begin{itemize}
        \item All expressions of System F are normalizable \cite{girard1972phd}.
        \item All expressions of System F are strongly normalizable \cite{PRAWITZ1971235}.
    \end{itemize}
\end{frame}

\plain{}{Bibliography}

\begin{frame}{Bibliography}
\bibliographystyle{amsalpha}
\bibliography{ref}
\end{frame}

\plain{}{Questions?}

